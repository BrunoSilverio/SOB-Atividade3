%%%%%%%%%%%%%%%%%%%%%%%%%%%%%%%%%%%%%%%%%
% University Assignment Title Page 
% LaTeX Template
% Version 1.0 (27/12/12)
%
% This template has been downloaded from:
% http://www.LaTeXTemplates.com
%
% Original author:
% WikiBooks (http://en.wikibooks.org/wiki/LaTeX/Title_Creation)
%
% License:
% CC BY-NC-SA 3.0 (http://creativecommons.org/licenses/by-nc-sa/3.0/)
% 
% Instructions for using this template:
% This title page is capable of being compiled as is. This is not useful for 
% including it in another document. To do this, you have two options: 
%
% 1) Copy/paste everything between \begin{document} and \end{document} 
% starting at \begin{titlepage} and paste this into another LaTeX file where you 
% want your title page.
% OR
% 2) Remove everything outside the \begin{titlepage} and \end{titlepage} and 
% move this file to the same directory as the LaTeX file you wish to add it to. 
% Then add \input{./title_page_1.tex} to your LaTeX file where you want your
% title page.
%
%%%%%%%%%%%%%%%%%%%%%%%%%%%%%%%%%%%%%%%%%
%\title{Title page with logo}
%----------------------------------------------------------------------------------------
%	PACKAGES AND OTHER DOCUMENT CONFIGURATIONS
%----------------------------------------------------------------------------------------

\documentclass[12pt]{article}
\usepackage[english]{babel}
\usepackage[utf8x]{inputenc}
\usepackage{amsmath}
\usepackage{xkeyval}
\usepackage{graphicx}
\usepackage[colorinlistoftodos]{todonotes}

\begin{document}

\begin{titlepage}

\newcommand{\HRule}{\rule{\linewidth}{0.5mm}} % Defines a new command for the horizontal lines, change thickness here

\center % Center everything on the page
 
%----------------------------------------------------------------------------------------
%	HEADING SECTIONS
%----------------------------------------------------------------------------------------

\textsc{\LARGE Pontifícia Universidade Católica Campinas}\\[1.5cm] % Name of your university/college
\textsc{\Large Engenharia de Computação}\\[0.5cm] % Major heading such as course name
\textsc{\large Sistemas Operácionais B}\\[0.5cm] % Minor heading such as course title

%----------------------------------------------------------------------------------------
%	TITLE SECTION
%----------------------------------------------------------------------------------------

\HRule \\[0.4cm]
{ \huge \bfseries Relatório Experimento 1}\\[0.4cm] % Title of your document
\HRule \\[1.5cm]
 
%----------------------------------------------------------------------------------------
%	AUTHOR SECTION
%----------------------------------------------------------------------------------------

\begin{minipage}{0.4\textwidth}
\begin{flushleft} \large
\emph{Author:}\\
Bruno Camilo \textsc{Silvério}
Guilherme \textsc{Pernicone}
João Pedro \textsc{Porta}
Marcelo Dib \textsc{Coutinho}
Pedro Pierina \textsc{Pierina}
\end{flushleft}
\end{minipage}
~
\begin{minipage}{0.4\textwidth}
\begin{flushright} \large
\emph{Supervisor:} \\
Ms. Prof. Edmar Roberto Santana de \textsc{Resende} % Supervisor's Name
\end{flushright}
\end{minipage}\\[2cm]

% If you don't want a supervisor, uncomment the two lines below and remove the section above
%\Large \emph{Author:}\\
%John \textsc{Smith}\\[3cm] % Your name

%----------------------------------------------------------------------------------------
%	DATE SECTION
%----------------------------------------------------------------------------------------

{\large \today}\\[2cm] % Date, change the \today to a set date if you want to be precise

%----------------------------------------------------------------------------------------
%	LOGO SECTION
%----------------------------------------------------------------------------------------

\includegraphics[width=26mm]{logo.jpg}% Include a department/university logo - this will require the graphicx package
 
%----------------------------------------------------------------------------------------

\vfill % Fill the rest of the page with whitespace

\end{titlepage}


\begin{abstract}
Experimento com o objetivo de aprender e entender as dificuldades de criar e colocar em uso um modulo de Kernel do sitema operácional Ubuntu 16.04.
\end{abstract}

\section{Introdução}

Nesse experimento, utilizando Ubuntu 16.04, foi criado um módulo que tem como objetivo utilizar a biblioteca crypto.h para cryptografar, decriptografar uma string dada pelo usuário atravéz de um programa de usuário, escrito em C, o programa de usuário comunica com o módulo atravéz da escrita e leitura no modulo /dev/cryptomodule 

\section{O que foi desenvolvido}

\subsection{Progama de usuário em C}

O programa em C possue um menu em que o usuário tem a opção de, encriptar uma string, encriptar escrevendo os bytes em hexadecimal, e decriptando utilizando tambem os valores dos bytes em hexadecimal, o programa então faz as escrita no módulo /dev/cryptomodule, inserindo no primeiro byte da string uma letra equivalente a operação a ser realizada, para então poder ler do mesmo módulo. O programa, também, se selecionado para encriptar um string comum, faz a conversão de uma string para um string com os valores originais representados em hexadecimal.

\subsection{Módulo}

O módulo, acionado após uma escrita em /dev/cryptomodule, recebe a string que o usuário proporcionou, e dependendo da primeira letra da string, faz a operação requisitada pelo o usuário, que pode ser, encriptação, desencriptação ou hash da string, e o modulo faz essas operações utilizando a biblioteca crypto.h.

\vfill

\section{Desafios Encontrados}

Durante o processo de desenvolvimento foram encontrados varios desafios:

\begin{enumerate}
\item Comunicação entre programa de usuário e módulo
\item Trasformação de uma string com caracteres hexadecimais para sua sua representação original
\item Utilização da biblioteca crypto
\item Scatterlist e como utiliza-las
\end{enumerate}

\subsection{Comunicação entre programa de usuário e módulo}

Essa dificuldade foi facilmente superada após estudo de códigos prontos disponibilizados pelo professor.
Utilizamos a função write e read em C, aprendemos sobre essas funções durante a execução do experimento, pois até o momento todos integrantes só havim utilizados as funções fwrite e fread, o conceito é parecido, entretanto fread e fwrite só são utilizados para arquivos.

\subsection{Trasformação de uma string com caracteres hexadecimais para sua sua representação original}

Para esse problema utilizamos manipulação de bits com a operação shifleft \textbf{\textless\textless}, e com com a subtração do caracter para o valor que ele representa utilizando os valores da \textit{ASCII Table}.
\begin{flushleft}
\begin{em}
Ex.:\\
4A      $\Rightarrow$ 00110100 01000001 : 2 caractéres\\
4Ah     $\Rightarrow$ 00000100 00001010 : Caractéres subtraídos para o valor original em hexadecimal\\
400Ah   $\Rightarrow$ 01000000 00001010 : O primero byte é shiftado 4 vezes\\
J / 4Ah $\Rightarrow$ 00101010          : Os dois bytes são somados, gerando o valor original que os dois caractéres representavam \\ 
\end{em}
\end{flushleft}

\subsection{Utilização da biblioteca crypto}
Esse problema não foi específico a uma função ou a uma lógica, mas sim a diferentes mecânismos necessários para a utilização do modulo que o tornou complicado de utilizar, foi possível sua utilização atravez de estudos dos códigos prontos e vários testes. 

\subsection{Scatterlist e como utiliza-las}
Scatterlist é uma struct em que são armazenadas \textit{pagenumber}, \textit{offset} e \textit{size} que são todas as informções para encontrar uma variável na memória, esse método de armazenar os dados de uma variável é muito utilizada em módulos de kernel.\\
Para resolver o problema de encontrar essa variável foi encontrado a função \textit{sg\_virt} que calcula o endereço da variável dando sua respectiva scatterlist, assim possibilitando a utilização dessa variável

\section{Conclusão}

\end{document}
